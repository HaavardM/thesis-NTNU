\chapter{Discussion}

All of the methods described in this thesis so far have their strong advantages and disadvantages. 

\section{When all variables are discrete}
In the case of \acrshort{pgm}'s with only discrete variables, exact inference methods are often applicable. The challenges of Bayesian inference in such networks usually boils down to computational complexity when the number of variables increases. Exact methods such as \acrshort{bp} are in most cases the most appropriate choice, though some structures (in the case of loops) cannot guarantee an optimal solution. 

\section{When some variables are continuous}
When some variables are continuous, the problem usually gets a bit more complicated. If all parent-child can be expressed using conjugate-priors, exact methods can still be used. However, requiring the use of only conjugate-priors may in many cases be too restrictive as it limits the ability to express intuitive understanding of the data-generating process. 

The most straight-forward approach will be to use \acrshort{mcmc} methods, due to their simplicity. These methods allow sampling from arbitrarily complex models as long as the target probability can be evaluated. It can in most cases provide asymptotic guarantees that the samples is from the true posterior distribution, though only given a very large amount of samples. How many samples that are considered "enough" is difficult to say, and it usually require manual interpretations of the results in order to verify convergence. Using \acrshort{mcmc} in autonomous systems may therefore be challenging unless the model is sufficiently simple, in which case other methods may still be preferable. Due to the random nature of sampling methods, \acrshort{mcmc} is likely a poor choice for problems where deterministic behaviour is valued.



Though it requires more work, \acrshort{vi} will in many cases be a better option. By posing the problem as a optimization problem rather than relying on sampling, variational inference can give deterministic behaviour as well as drastically speed up inference when compared to \acrshort{mcmc}. However, \acrshort{vi} requires manual selection of a good surrogate density, and the choice of a bad surrogate may lead to poor results due to invalid assumptions. \acrshort{vi} does therefore by itself not provide any guarantees on the correctness of the results as it will always be limited by the assumptions and approximations used by the surrogate density.

A lot of research is currently focusing on how to apply \acrshort{vi} on more complicated models without the need of error-prone calculations or strict assumption. Methods such as variational message passing allows for more complicated surrogate densities in order to retain the interaction between variables in the model \cite{winnbishop}. 

In practice, both methods are likely needed. \acrshort{mcmc} methods can be used to "blindly" sample from the true posterior in order to aid the selection of a proper surrogate density. \acrshort{vi} can then be used to approximate the true posterior without loosing information to invalid assumptions. 

\section{}

