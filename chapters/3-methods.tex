\chapter{Survey of Training Methods}

\section{Markov-Chain-Monte-Carlo}
The main idea behind Markov-Chain-Monte-Carlo (MCMC) algorithms is to construct Markov Chains whose stationary distribution is the desired target density. After drawing many correlated samples from such a chain, the fraction of time spent in each state will be proportional to the target density $p^*(\mathbf{x})$ \cite{murphy}. 

\subsection{Metropolis Hastings Algorithm}
The Metropolis Hastings Algorithm (MH) is described in \cite[p.~850]{murphy}.
For each current state $\mathbf{x}$, a new state $\mathbf{x'}$ is proposed according to a proposal density $q(\mathbf{x'} | \mathbf{x})$. The proposal density can be chosen freely as long as it gives a non-zero probability of moving to states with non-zero probability in the target density. In other words, $q(\cdot)$ must cover all possible realizations of $p^*(\cdot)$. 

After proposing a move to $\mathbf{x'}$, the proposal is either \textbf{rejected} or \textbf{accepted} according to a formula which ensures that the stationary distribution of the Markov chain equals the target distribution. 

For a \textbf{symmetric} proposal distribution, i.e. $q(\mathbf{x'} | \mathbf{x}) = q(\mathbf{x}| \mathbf{x'})$, the acceptable probability is given by 
\begin{equation}
    r = \min(1, \frac{p^*(\mathbf{x'})}{p^*(\mathbf{x})})
\end{equation}

For an \textbf{asymmetric} proposal, i.e. $q(\mathbf{x'} | \mathbf{x}) \neq q(\mathbf{x} | \mathbf{x'})$, the \textit{Hastings correction} is needed to compensate for any bias introduced by $q(\cdot)$
\begin{subequations}.
\begin{align}
    r &= \min(1, \alpha)\\
    \alpha &= \frac{p^*(\mathbf{x'}) q(\mathbf{x'} | \mathbf{x})}{p^*(\mathbf{x}) q(\mathbf{x}| \mathbf{x'})}\label{eq:mcmc_mh_acceptance}
\end{align}
\end{subequations}

The normalization constant in the target density $p^*(\cdot)$ will cancel in equation \eqref{eq:mcmc_mh_acceptance}. The MH algorithm can therefore be used with unnormalized distributions, making it a very useful algorithm for sampling from complex distributions. 

A common proposal density is a symmetric Gaussian distribution centered at the current state, i.e. $$q(\mathbf{x'} | \mathbf{x}) = \mathcal{N}(\mathbf{x'} | \mathbf{x}, \boldsymbol{\Sigma})$$ resulting in the \textit{random walk Metropolis Algorithm}.

If the proposal density $q(\cdot)$ is independent of the current state, i.e. $$q(\mathbf{x'} | \mathbf{x}) = q(\mathbf{x'})$$ then the MH algorithm boils down to the \textit{independence sampler}, which is similar to importance sampling as described in \cref{sec:theory_importance_sampling} \cite{murphy}.

The MH algorithm is summarized in \cref{alg:metropolis_hastings}.
\begin{algorithm}\label{alg:metropolis_hastings}
\SetAlgoLined
\For(){t = 0, 1, 2}{
    Sample $x' \sim q(x' | x_t)$ \;
    Compute acceptance probability \\
    $\alpha = \frac{p^*(\mathbf{x'}) q(\mathbf{x'} | \mathbf{x}_t)}{p^*(\mathbf{x}_t) q(\mathbf{x}_t | \mathbf{x'})}$\;
    Compute $r = \min(1, \alpha)$\;
    Sample $u \sim \mathcal{U}(0, 1)$\;
    Set $x_{t+1} = \begin{cases}x' & \text{if } u \leq r\\x_t & \text{otherwise}\end{cases}$;
}
\caption{Metropolis Hastings Algorithm}
\end{algorithm}

\section{Expectation Maximization}

\section{Variational Bayes}

\section{A Note on Probabilistic Programming}

