\chapter*{Abstract}

Autonomous ships depend on situational awareness in order to avoid collisions in a safe and robust manner. By knowing the intention of surrounding vessels, safety margins can be improved by avoiding situations with increased risk. In this thesis, methods for Bayesian Inference will be explored, with the goal of developing a flexible framework for intention modelling. Exact and approximate inference methods are explored. Approximate methods are found to be more flexible, allowing easier incorporation of existing knowledge from domain experts or conventions such as \Gls{colregs}. Methods such as \acrfull{mcmc} and \acrfull{vi} are therefore explored further and compared on an illustrative intention model. The results then find \acrshort{mcmc} to be accurate at the cost of computational complexity. \acrshort{vi} on the other hand, is found to be a lot faster, though much less precise on the illustrative model. 