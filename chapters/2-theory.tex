\chapter{Neccessary Theoretical Background}

\section{Useful Results From Probability Theory}
 
\subsection{Conditional Probabilities}
The \texttt{conditional probability} $p(A | B)$ is the probability of $A$ occurring, given the known occurence of another event $B$. This can be interpreted as knowing the value of $B$ includes some information about $A$. Mathematically it can be expressed as
\begin{equation}\label{eq:conditional_probability}
    p(A | B) = \frac{p(A, B)}{p(B)}
\end{equation}

\subsection{Bayes Law}

A useful extension to \cref{eq:conditional_probability} is to recognize that the joint distribution $p(A, B)$ can be rewritten as a product of a conditional probability $p(B | A)$ and $p(A)$. Inserting into \cref{eq:conditional_probability} yields \texttt{Bayes Law}
\begin{equation}\label{eq:bayes_law}
    p(A | B) = \frac{p(A, B)}{p(B)} = \frac{p(B | A)p(A)}{p(B)}.
\end{equation}

As $B$ is known the denominator $p(B)$ is simply a normalizing constant. It is sometimes useful to rewrite \cref{eq:bayes_law} as
\begin{equation}\label{eq:bayes_law_proportional}
    p(A | B) = \frac{p(B | A) p(A)}{p(B)} \propto p(B | A)p(A)
\end{equation} 
which is useful if $p(B)$ is hard to calculate and we do not need normalized values of $p(A | B)$.

\subsection{Marginal Probability \& The Law of Total Probability}
The marginal probability of an event $X$ is the probability of $X$ occuring irrespective of any other variables.
For notational simplicity we use the integral operator for marginalization of both continuous and discrete random variables, even though the integral is replaced by a sum for discrete random variables. For an event $X$ and any other variables $\bf Y$, the marginal probability of $X$ can be written as
\begin{equation}
    p(X) = \int_\mathbf{Y} p(X, \mathbf{Y}) d\mathbf{Y} = \int_\mathbf{Y} p(X | \mathbf{Y}) p(\mathbf{Y}) d\mathbf{Y}
\end{equation}
The last equation is by the \texttt{Law of total probability} which relates the marginal probabilities to conditional probabilities. This is a result we will use extensively as it allows us to split complex distributions into more managable pieces.

\subsection{Interpreting Probability}
The results we have mentioned so far stems from abstract mathematical axioms, and do not tell us how to interpret the resulting probabilities. There are however different interpretations which are commonly accepted. Perhaps the two biggest are the frequentist and bayesian interpretations. 

\begin{description}
    \item[The Frequentist Interpretation:] The frequentists defines an events probability as the limit of it's relative frequency over many trials. In other words, the probabilities are assigned a physical interpretation and remains rather objective. There do however arise issues and paradoxes when we try to assign probabilities to events which are not recurrent, i.e. they only happens a few times. The goal of frequentists is to either reject or accept a hypothesis by evaluting the likelihood of the data $D$ given a model $M$, i.e. $p(D | M)$. 
    \item[The Bayesian Interpretation:] The bayesians interprets probability as a state of knowledge \cite{Jaynes86bayesianmethods:}. Data is used to update our prior knowledge about an event, and the probabilities is used to quantify how strongly we believe in each outcome. This interpretation is highly philosophical, but beautifully captures humans intuitive reasoning. The goal of bayesian statistics is to evaluate the likelihood of the model $M$ given the observed data $D$, i.e. $p(M | D)$, usually expressed through Bayes law (\cref{eq:bayes_law}). The bayesian interpretation do however involve a level of subjectivity when choosing priors, making it difficult to form an objective opinion from data. For those interested, see \Cite{Jaynes86bayesianmethods:} for a fascinating read on the history of Bayesian Probability.
\end{description}

While the differences between the frequentist and bayesian interpretations are mostly philosophical, there are a few practial differences. For a frequentist it do not make sense to talk about any probabilities before an experiment has been performed. The prior $p(M)$ and posterior $p(M | D)$ is therefore nonsencical and cannot be computed using a frequentist interpretation.     





\section{Bayesian Statistics}

Using \cref{eq:bayes_law} we can write 
\begin{equation}\label{eq:bayes_learning}
    p(A|BC) = \frac{p(B | AC) p(A | C)}{p(B | C)} \propto p(B | AC)p(A | C)
\end{equation}

If $A$ is the unknown outcome we are interested, $B$ represents data we have collected, and $C$ is any prior knowledge we have about $A$, then \cref{eq:bayes_learning} is a mathematical representation of the process of learning \cite{Jaynes86bayesianmethods:}. The variable $A$ usually represent some sort of hypothesis we want to verify. 

The factors of \cref{eq:bayes_learning} can be interpreted as
\begin{description}
    \item[The Prior $p(A | C)$:] The prior is our knowledge about $A$ before observing any data. This can be domain-specific knowledge, results from prior experiments or intuitive reasoning about possible values of $A$. 
    \item[The Likelihood $p(B | AC)$]: The likelihood of $B$ tells us how likely it is to observe the data $B$ given that $A$ is true. In other words, how well does the observations $B$ fit with our beliefs ($A$). 
    \item[The Posterior $p(A | BC)$]: The posterior distribution is our updated belief about $A$ after observing $B$. This is our knowledge about $A$ after observing the data. 
\end{description}

\section{Stochastic Modelling}

\section{Probabilistic Graphical Models}

\section{Belief Propagation}

\section{Markov Chains}

