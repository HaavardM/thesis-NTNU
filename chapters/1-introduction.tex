\chapter{Introduction}

Humans, while resourceful, are prone to loss of focus, tiredness and have limited attention span. Human errors are estimated to contribute to more than $75\%$ of maritime accidents \cite{Tengesdal2020RiskbasedAM}. On the contrary, computers always remain "focused" and never gets tired. Using autonomus ships as a replacement for humans can greatly reduce the number of maritime accidents.
While completely avoiding humans at sea would increase safety, it simply is unrealistic. Autonomous ships will need cooperate with humans on nearby vessels to avoid dangerous situations. Robust collision avoidance systems (COLAV) will be necessary to  maneuver in areas with potential obstacles in order to avoid collisions. However, the optimal decision is usually highly situation-dependent and requires ability to understand and predict the traffic patterns of nearby vessels, either autonomous or controlled by humans.  

Teaching machines human behavior can be difficult. A human steering a ship has the ability to somehow predict the intention of nearby vessels by simply observing and compare the behaviour to own experiences. Rules and conventions such as COLREGs \cite{colreg} also affects the behaviour of especially larger vessels, but are not necessarily followed exactly. Some situations require rules to be broken, in order to avoid collisions or other dangerous situations. 

Bayesian Networks, often called Belief Networks, are Probabilistic Graphical Models represented as directed acyclic graphs \cite{murphy}. There is strictly speaking nothing Bayesian about Bayesian Networks (they may just as well be used with Frequentist statistics), as it is simply a way to describe probability distributions \cite{murphy}. Bayesian Networks are also often used to represent causal relations, though the interactions are in no way required to be causal. However, the causal interpretation of Bayesian Networks allow humans to intuitively understand relations between variables. Bayesian Networks are for this reason heavily used in Causal Inference, which attempts to model and infer true causal relationships between variables \cite{causal}. Bayesian Networks intuitive structure further allows human domain experts to reason about and participate in the development of statistical models, without requiring deep statistical knowledge. 
This makes Bayesian Networks a good choice for intention models, as the model need to incorporate prior knowledge such as intuitive reasoning and predefined rules (such as COLREGS), as well as be able to learn from new and prior observations. 

This thesis will investigate how inference can be performed from available data in a Bayesian Network. At this point the goal is not to find a realistic intention model, but rather try to develop a flexible framework for inference in a known model. Exact methods using Belief Propagation will be discussed, but is found to only be feasible for a few types of models. Sampling based methods such as \acrshort{mcmc} are then explored as a approximate solution to Bayesian inference in arbitrary models, though at the cost of computational complexity and stochastic behaviour. Variational Bayes is then explored in order avoid the computational complexity of \acrshort{mcmc}, while still be able to apply Bayesian Inference in complex models.  

This thesis is divided up in multiple chapters. Some neccessary theoretical background, mainly an introduction to Bayesian Statistics and \acrfull{pgm}'s, are presented in \cref{chap:theory}. The theory for \acrshort{mcmc} based methods are then found in \cref{chap:mcmc}, while excact and approximate analytical methods are found in \cref{chap:analytical}. In order to not only present the theoretical concepts, implementations of \acrshort{mcmc} and \acrshort{vi} are compared on a fictitious model in \cref{chap:impl}. Finally, the theory and results for the different methods are then compared and discussed in \cref{chap:discussion}.