\chapter{Introduction}

 Teaching machines human intution and behavior can be difficult. A human steering a ship has the ability to somehow predict the intetion of nearby vessels by simply observing and compare the behaviour to prior experiences. Rules and conventions such as COLREGs \cite{colreg} also affects the behaviour of larger vessels, but are not neccessarily followed excactly. Some situations may also require rules to be broken, in order to avoid collitions or other dangerous situations. 
 
 The Autoferry Project is a cross disciplinary research project at NTNU, with the end goal of developing autonomous, all-electric passanger ferries for urban water transport \cite{autoferry}. As these passenger ferries need to work in existing harbors alongside ordinary vessels, the understanding of existing traffic is critical in order to design safe and reliable autonomous ferries.
 
 Ongoing research in the Autoferry Project focus on developing robust and efficient Collision Avoidance (COLAV). \cite{Tengesdal2020RiskbasedAM} introduces a risk-based COLAV system which incorporates obstacle intention predictions. The paper focuses on how probabilistic intention information can be incorporated if available, and propose the use of Bayesian Networks in order to infer intentions from available data. The paper only demonstrate the effects through an illustrative example, and do not further discuss how the intention probabilities would be infered in practice.

 Bayesian Networks, often called Belief Networks, are Probabilistic Graphical Models represented as directed acyclic graphs \cite{murphy}. There is strictly speaking nothing Bayesian about Bayesian Networks (they may just as well be used with Frequentist statistics), as it is simply a way to describe probability distributions \cite{murphy}. Bayesian Networks are also often used to represent causal relations, though the interactions are in no way required to be causal. However, the causal interpretation of Bayesian Networks allow humans to intuitively understand relations between variables. Bayesian Networks are for this reason heavily used in Causal Inference, which attempts to model and infer true causal relationships between variables \cite{causal}. Bayesian Networks intuitive structure further allows human domain experts to reason about and participate in the development of statistical models, without requring deep statistical knowledge. 
 
 This thesis will investigate how inference can be performed from available data in a Bayesian Network. At this point the goal is not to find a realistic intention model, but rather try to develop a flexible framework for inference in a known model. Excact methods using Belief Propagation will be discussed, but is found to only be feasible for a few types of models. Approximate methods such as \acrfull{mcmc} and \acrfull{vi} are then explored as alternatives which allows inference in more flexible models, though at the cost of runtime and precision. 

 This thesis is divided up in multiple chapters. Some neccessary theoretical background, mainly an introduction to Bayesian Statistics and \acrfull{pgm}'s, are presented in \cref{chap:theory}. The theory for \acrshort{mcmc} based methods are then found in \cref{chap:mcmc}, while excact and approximate analytical methods are found in \cref{chap:analytical}. In order to not only present the theoretical concepts, implementations of \acrshort{mcmc} and \acrshort{vi} are compared on a fictitious model in \cref{chap:impl}. Finally, the theory and results for the different methods are then compared and discussed in \cref{chap:discussion}.