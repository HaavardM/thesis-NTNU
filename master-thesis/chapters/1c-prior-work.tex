\chapter{Previous Work}

There is currently limited work available on using \acrshort{gp}s on \acrshort{ais} data.

\citeauthor{gpanomaly} propose a method for anomaly detection of moving vessels in \cite{gpanomaly}. The approach uses a \acrshort{gp} to learn vessels' normal behavior in an area, which is then used to detect abnormal behavior. The work further addresses the computational challenges when using large datasets and proposes an active learning approach that iteratively adds new training samples to select the optimal training sample. A relatively small sample size is then used to represent the entire dataset. Active learning is computationally feasible by updating the Cholesky decomposition at each step instead of recomputing the entire decomposition. While the approach works well for anomaly detection, it was not intended for predicting future trajectories, only to classify existing trajectories.

\citeauthor{gp_ais_trajectory} address the problem of trajectory uncertainty in \cite{gp_ais_trajectory}. The paper proposes a method for probabilistic trajectory prediction using \acrshort{gp}s, where different distributions describe the lateral and longitudinal directions of the trajectory. The parameters of the distributions are learned offline based on historical \acrshort{ais} data. They are then applied in real-time by adding new observations through a Cholesky update to avoid recomputing the decomposition. The model uses the common \acrshort{rbf} kernel, and the hyperparameters are selected as the median of the maximum likelihood estimates for each unique vessel. The prediction is then conditioned on historical \acrshort{ais} samples for a given vessel.
A case study is performed where the model is trained and tested on three months of \acrshort{ais} data along a smooth traffic lane. The methods yield high prediction accuracy even for long prediction horizons while still applying to real-time applications. Only a highly smooth traffic lane with little curvature is used, and the paper does not mention the performance on curved trajectories. As the model is only conditioned on previous samples of a given vessel, it is unlikely that the model is able to predict upcoming turns. 

There is, however, extensive work on using \acrshort{gp}s for trajectory prediction outside of the maritime field. \citeauthor{vehicle_gp_prediction} in \cite{vehicle_gp_prediction} utilizes \acrshort{gp}s for long-term trajectory prediction for collision avoidance in a connected vehicle environment. The vehicles are assumed to share position through vehicle-to-vehicle communication, somewhat similar fashion to \acrshort{ais} for maritime vessels. The paper uses a \acrshort{gp} to learn a motion model from historical data, mapping a vehicle's current position to a trajectory derivative. The historical data is clustered into a finite number of clusters, where the trajectories in each cluster are assumed to have similar properties. The paper utilizes K-means clustering \cite{murphy} to group trajectories based on the first and last position of a trajectory. For each cluster, independent \acrshort{gp}s are then fitted for each of the two coordinate axes, using independent \acrshort{rbf} kernels and zero mean priors. While this method should be applicable to \acrshort{ais} data, there are a few key distinctions to keep in mind:
\begin{enumerate}
    \item In this paper, the trajectory derivatives are calculated using finite difference with a sampling interval of $0.1$ seconds. The typical sampling interval for \acrshort{ais} data is in the range of minutes, so relying on numerical derivatives might be challenging.
    \item The vehicles trajectories are from an intersection, and the roads constrain the vehicles' behavior. There is, therefore, only a finite number of route options a vehicle may take.
\end{enumerate}

A similar approach is used by \citeauthor{pedestrian} in \cite{pedestrian} to predict the trajectory of pedestrians tracked using computer vision. The paper also learns a dynamical model using \acrshort{gp}s but utilizes a Bayes Filter framework proposed by \citeauthor{gpekf} in \cite{gpekf} to simulate the trajectories. The trajectories were then simulated using two different approaches:
\begin{enumerate}
    \item By assuming $p(\boldsymbol{x})$ is always uni-modal and Gaussian, the GP-EKF introduced in \cite{gpekf} was used to simulate the trajectory for multiple timesteps, using the dynamical \acrshort{gp} model as the prediction model. The GP-EKF is based on the Extended Kalman Filter, where the prediction model is learned from data using a \acrshort{gp}. This formulation is unable to express multimodal uncertainty.
    \item To retain the inherent multimodality, a sequential Monte-Carlo approach (i.e., the prediction step of a particle filter) was used to keep track of multiple modes (i.e., branching trajectories) at the cost of computational complexity.  
\end{enumerate}





There have been several attempts on utilizing Big Data \acrshort{ais} to improve long-term vessel trajectory prediction. Most of these method is based on clustering methods.

\subsection{Clustering-based method}
These methods typically work by clustering historical \acrshort{ais} data and then create representable paths to use when predicting vessels' future trajectories. 


Traditional clustering methods, such as k-means or DBSCAN, tend to focus on clustering point values. In the context of trajectory prediction, the trajectories would then be clustered as a whole. The trajectory clustering algorithm TRACLUS was therefore introduced by \cite{traclus} where it was applied to hurricane trajectory and animal movement data. The key observation was that trajectories might have portions that share common behavior, while the entire trajectory might still differ. TRACLUS allows clustering trajectories based on common sub-trajectories and works by partitioning the trajectories into smaller line segments and then groups similar segments into clusters.  
