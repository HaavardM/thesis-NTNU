\chapter{Learning trajectories directly from data}
As a first attempt on using \acrshort{gp}s for trajectory prediction, a direct approach was attempted. The idea was to use a \acrshort{gp} to model a function $f: \mathcal{R}^5 \to \mathcal{R}^2$, mapping a ships current position $\boldsymbol{x}_0$, \acrshort{cog} $\psi_0$ and \acrshort{sog} $v_0$ to a position at time $\tau$. The method is formulated in mathematical terms in \cref{eq:gp_direct}. 

\begin{subequations}\label{eq:gp_direct}
\begin{align}
    \boldsymbol{f}(\boldsymbol{\eta}) &= \boldsymbol{x}_{\tau} \label{eq:gp_direct_f}, \quad \boldsymbol{\eta} = \begin{bmatrix} \boldsymbol{x}_0 & \psi_0 & v_0 & \tau\end{bmatrix}\\
    \boldsymbol{f}(\boldsymbol{\eta}) &\sim \text{GP}(\boldsymbol{m}(\boldsymbol{\eta}), K(\boldsymbol{\eta}, \boldsymbol{\eta}))\label{eq:gp_direct_f_dist}
\end{align} 
\end{subequations}

The model could then be queried about likely future positions by evaluating $\boldsymbol{f}$ with the desired initial conditions and time horizon. 

The conceptual benefits of this formulation include:
\begin{description}
    \item[Continuous formulation] The model directly models the position at any time $t+\tau$ and does not require discretization. 
    \item[Conceptually easy] This formulation directly expresses the unknown trajectory, making it easy to reason about conceptually.
    \item[Simple Problem formulation] The model requires few components.
    \item[Easy incorporation of available AIS data] The reported \acrshort{cog} and \acrshort{sog} can easily be incorporated into the similarity measure, utilizing more of the available data.
\end{description}

However, this formulation was impractical to work with and was abandoned after several attempts to get it to work. The attempted method and the issues which followed are discussed in this chapter.


\section{Method}
While simple, the formulation in \cref{eq:gp_direct} still lead to quite a lot of complexity due to the number of input parameters. A large number of input dimensions requires large amounts of data, which is infeasible using the exact \acrshort{gp}. Two different approximations were instead attempted and will be discussed in this section.

\begin{enumerate}
    \item Selecting only a subset of the data which is relevant for a given prediction. Based on the initial conditions in $\boldsymbol{x}$, a subset of representative trajectories can be selected and used for training. The exact \acrshort{gp} formulation in \cref{alg:gp_prediction} can then be used, though it requires both hyperparameter tuning and recalculating Cholesky decomposition $L$ for each query input $x$.
    \item Approximating the \acrshort{gp} using the \acrshort{svgp}, the entire dataset can be used at the cost of increased model complexity. It is also an approximation method, so there are a few tradeoffs when using \acrshort{svgp}. The benefit is that a \acrshort{gp} can be trained once for a large area and be used for multiple predictions.
\end{enumerate}

\subsection{Excact GP using only a subset of the data}
The kernel matrix $K$ for all available data will be approximately sparse. Only a subset of the \acrshort{ais} data is actually relevant for any given query. Trajectories with very different initial conditions than the query input should be safe to ignore, leaving only a subset of relevant trajectories to use for training. This way, we can still use the exact \acrshort{gp} formulation and avoid the additional complexity of \acrshort{gp} approximations. The following requirements need to be satisfied for the trajectories used to train the \acrshort{gp}:
\begin{enumerate}
    \item The initial position of the trajectory must be close to the queried position $\boldsymbol{p}_0$.
    \item The initial \acrshort{cog} must be close to the queried heading $\mathcal{X}_0$.
    \item The initial \acrshort{sog} must be close to the queried velocity $v_0$.
\end{enumerate} 

The excact \acrshort{gp} formulation from \cref{chap:theory} can then be applied, optimizing \cref{eq:gp_log_marginal_likelihood} to perform model selection and using \cref{alg:gp_prediction} to perform predictions. 

\subsection{Sparse Variational Gaussian Process}
While filtering the data before making a prediction works well for isolated cases, it does not scale well. The model needs to be retrained for each query, which results in a significant performance penalty in practical applications. As an alternative, we can attempt to use the \acrshort{svgp} instead to train a single \acrshort{gp} which can be used for any queries in an area. The training time will be a lot longer but can be performed in advance. 

\section{Choice of kernels}

We attempted multiple types of kernels. 

\section{Implementation}
The excact \acrshort{gp} formulation was implemented using the \texttt{GaussianProcessRegressor} from the popular Python library, \textit{sciki-learn} \cite{scikit-learn}. The library support all kernels introduced in \cref{chap:theory} and supports hyperparameter optimization using multiple restarts to avoid bad local optimas. 

For the \acrshort{svgp} implementation, the Python library \textit{GPFlow} \cite{GPflow2017} built on the well-known \textit{tensorflow} \cite{tensorflow2015-whitepaper} library was used. GPFlow has several implementations of approximate \acrshort{gp}s, including the acrshort{svgp}. For optimization, the stochastic gradient descent optimizer \textit{ADAM} should be familiar for anyone working with Neural Networks. This optimizer was selected to optimize both the inducing variables as well as the hyperparameters through \textit{mini batching}. Mini Batching is an optimization technique used in combination with stochastic optimization, where a random subset of the training set is used in each iteration instead of the full dataset. This way, big datasets can be used to train the model.\todo[]{This is from memory, find a source}

\subsection{Choice of kernel}


\section{Results}

In practice, this formulation turned out to be challenging to work with, primarily due to the increased complexity of \acrshort{svgp} and a large number of input dimensions.

\subsection{Complexity}
Building the model using \acrshort{svgp} turned out to be challenging. While the model sometimes was able to achieve reasonable \acrshort{mse} given enough training iterations, the trajectories were by visual inspection found to be unrealistic. The high complexity of the model made it challenging to pinpoint the underlying cause of prediction errors. The benefit of \acrshort{gp}s interpretability is lost when using these approximate methods.

\subsection{Optimization}
The optimization would often result in unrealistic hyperparameters, such as prioritizing \acrshort{cog} and \acrshort{sog} over the position. This would yield predictions where the initial position was far away from the queried position. This problem could potentially be fixed by specifying priors for the hyperparameters, but this would also further increase the complexity. 

Optimizing both the hyperparameters and the inducing variables was extremely slow and required hours to get good results. Due to the high number of input dimensions, a large number of inducing variables were required to get a good approximation. 




