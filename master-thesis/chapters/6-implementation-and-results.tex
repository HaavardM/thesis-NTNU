\chapter{Statistical Testing}\label{chap:stat_testing}
Inspired by \cite{hexeberg}, the performance of the proposed models will be compared on straight-line and curved trajectories independently. This comparison is intended to showcase the methods' abilities to consistently predict reasonable trajectories and motivate further research into \acrshort{gp} based methods. The content of this chapter is not intended to showcase a perfect solution, and the methods are still heavily influenced by how the hyperparameters are selected and any tuning parameters.   

\section{Method}

The simulation will test the performance on a total of $700$ different test trajectories, where one half corresponds to straight-line trajectories and the remaining half is curved trajectories. Trajectories are selected from the dataset proposed in \cref{chap:ais}, and will have a total duration of between $15$ and $30$ minutes. The Direct GP and GP-EKF from \cref{chap:direct_gp} and \cref{chap:gp_ekf} respectively will be tested on the same scenarios as defined in \cref{chap:ais}, getting access to the excact same training data. 

Three different metrics will be used to compare the performance of the different methods. 

\subsection{Trajectory Error}
The trajectory error is found by comparing the predicted mean position with the ground truth. As the predicted trajectory is simulated in discrete time, the points with the closest timestamps are used for comparison. The simulation will use $\Delta \tau = 10\text{ seconds}$, which yields a maximum error in time $\frac{\Delta T}{2} = 5 \text{ seconds}$, which is considered to be acceptable considering the time-horizon of between $15$ and $30$ minutes.
\subsection{Path Error}
The path error is defined as the closest point in the predicted trajectory to each point in the ground truth, under the constraint that the corresponding predicted timestamps must be monotonically increasing. In other words, the path cannot move backward in time. Linear interpolation is used to get the path error at fixed timestamps to simplify the comparison.

\subsection{Normalized Estimation Error Squared}
For the uncertainty estimates to provide any value, the predictions must be consistent. In this context, the term consistency is borrowed from the term \textit{filter consistency} used when tuning Kalman filters \cite{sensorfusjon}. The idea is that prediction errors, on average, should scale with the state covariance. In other words, the model should not place much confidence in a prediction that is wrong while being highly confident when a prediction is correct. Consistency can also be interpreted using a frequentistic interpretation of probability, where after many predictions, the state uncertainty should reflect the actual error rate.

The \textit{\acrfull{nees}} is a metric that can be used to quantify consistency and is given by
\begin{equation}
    \text{NEES} = (\boldsymbol{x} - \hat{\boldsymbol{x}})^\intercal \boldsymbol{P}^{-1} (\boldsymbol{x} - \hat{\boldsymbol{x}})
\end{equation}

Assuming the prediction error follows a Gaussian distribution, the \acrshort{nees} follows a Chi-Squared distribution which can be used to form a confidence interval. Comparing the prediction errors with this confidence interval can then be used to get a sense of whether the estimated state uncertainty is consistent with the actual error rate.

\subsection{Interpolation}
The metrics will be compared at fixed $5$ minute intervals using linear interpolation. The error for short trajectories is not extrapolated, so there might be fewer available samples for the metrics when moving past $15$ minutes.

\subsection{Baseline - Constant Velocity Model}
As a basis of comparison, the \textit{\acrfull{cvm}} method is used as a baseline. The model uses the initial \acrshort{cog} and \acrshort{sog} to predict a straight line, where the vessel is assumed to keep a constant velocity and heading.

\subsection{Selecting test cases}
The two datasets from \cref{chap:ais} are used to perform the statistical testing.

However, $54$ of the dataset originates from only three ferries. Distinct discontinuities characterize the trajectories of these ferries as the vessels dock for short periods before moving back the same way they came. These discontinuities directly contradict the assumptions of smoothness made by the \acrshort{gp}s used in this thesis. Whether or not to include these vessels was a challenging decision, as a \acrshort{colav} will need to be able to handle ferries, just as any other vessel type. However, in a practical application, it is natural to distinguish between different vessel types and allow the use of more specialized models to handle cases such as frequently docking ferries. As no such distinction is made in this thesis, it was decided that the statistical testing would be more insightful without using edge-cases for more than $50\%$ of the test cases. While it would be possible to not completely remove these vessels, i.e., perform subsampling, it was considered more confusing. Therefore, the samples from these three vessels are removed from the test set, though they are still present in the training set. This thesis has not proposed any method for filtering the training data based on the \acrshort{mmsi}; removing these vessels from the training set would be an unfair advantage.

The result is an artificial bias favoring the \acrshort{gp} framework proposed in this thesis, as the fraction of smooth trajectories has been artificially increased.

\subsubsection{Straigth-line trajectories}
The methods are first compared to the \acrshort{cvm} on simple straight-line trajectories. The statistics are based on $350$ randomly sampled trajectories without a replacement that satisfies the following requirements:
\begin{enumerate}
    \item The sum of subsequent changes in \acrshort{cog} must be less than $30$ degrees, i.e. $\sum_i |(\mathcal{X}_{t+1} - \mathcal{X}_t)| \leq 30^\circ$. This requirement ensures a straight-line trajectory. Additional care is always taken to use the smallest possible difference as the angles $\mathcal{X} \in [0^\circ, 360^\circ)$ wraps around.
    \item There must be sufficient data available for training in the neighborhood around the initial starting point, with similar initial heading and speed. After sanitizing the dataset and removing irrelevant trajectories, at least $3$ trajectories need to be available for training.
    \item The overall duration of the trajectories must be between $15$ and $30$ minutes to be considered a test candidate.
\end{enumerate}

\subsubsection{Curved Trajectories}
The curved trajectory statistics are based on $350$ randomly sampled trajectories that satisfy the following requirements:
\begin{enumerate}
    \item The sum of subsequent changes in \acrshort{cog} must be greater than $40$ degrees, i.e. $\sum_i |(\mathcal{X}_{t+1} - \mathcal{X}_t)| \geq 40^\circ$. This requirement ensures a minimum curvature in the trajectory. Additional care is always taken to use the smallest possible difference as the angles $\mathcal{X} \in [0^\circ, 360^\circ)$ wraps around.
    \item There must be sufficient data available for training in the neighborhood around the initial starting point, with similar initial heading and speed. After sanitizing the dataset and removing irrelevant trajectories, at least $3$ trajectories need to be available for training.
    \item The overall duration of the trajectories must be between $15$ and $30$ minutes to be considered a test candidate.
\end{enumerate}

\subsection{Training Data}
The initial conditions for each test trajectory is considered a distinct scenario, and the relevant training data is selected according to the method proposed in \cref{chap:ais}. The requirements are restated here with the concrete values used for testing:

\begin{enumerate}
    \item The trajectories' initial position of the trajectory must be close to the target vessel's position $\boldsymbol{x}_0$. In this thesis, a fixed threshold at $||\Delta \boldsymbol{x}_0|| \leq 200 \text{ m}$ is used, where $\Delta \boldsymbol{x}_0$ is the difference between the target vessel's current position and the initial conditions of a potential training trajectory.
    \item The trajectories' initial \acrshort{cog} must be close to the target vessel's heading $\mathcal{X}$. In this thesis, a fixed threshold at $\mathcal{X} \pm 20^\circ$ is used, with additional care taken when the angles wrap.
    \item The trajectories' initial \acrshort{sog} must be close to the queried velocity $v$. In this thesis, a fixed threshold at $v \pm 4 \text{ knots}$ is used.
\end{enumerate}

Due to the way trajectories are generated from the \acrshort{ais} dataset, there will be significant overlaps between trajectories. Naively dividing the trajectories into a train and test is therefore considered insufficient, as subtrajectories from the test set might also exist in the training set. Instead, the entire dataset is available for training, but all trajectories with identical MMSI and date as the test trajectory are removed before training. The date requirement ensures that trajectories for the same vessel can be used for training on any other day. There is still the possibility of leaking data from the test set if the trajectory moves past midnight, but this is assumed only to affect a negligible number of trajectories.


\section{Implementation}
The implementation details for each of the methods are described in this section.

\subsection{GP-EKF}
The GP-EKF is tested with and without the both the \acrshort{sl} and the \acrshort{pdaf} update steps, as well as using both finite difference and the \acrshort{cog}/\acrshort{sog} from the \acrshort{ais} dataset, resulting in $6$ different instances.

The GP-EKF required more flexibility during development. Due to the need for calculating the gradient $\frac{\delta \vec{f}}{\delta \boldsymbol{x}}$, it became impractical to use existing \acrshort{gp} implementations. Due to the simple implementation of \cref{alg:gp_prediction}, it was easier to implement it from the ground up.
The GP-EKF was therefore implemented directly in Python using only \textit{scipy}\cite{scipy} and \textit{numpy}\cite{numpy} to speed up linear algebra routines. The Cholesky decomposition in \cref{alg:gp_prediction} was computed using \texttt{scipy.linalg.cho\_factor}, which calls a highly optimized LAPACK routine. Similarily, \texttt{scipy.linalg.solve\_\-trianglular} was used to solve the lower triangular system of equations by forward substitution. The implementation of GP-EKF and the \acrshort{pdaf} update was then straightforward using \cref{alg:gp_ekf_prediction} and \cref{alg:gp_ekf_pdaf} from \cref{chap:gp_ekf}. The training outputs $\boldsymbol{y}$ was standaridized.

For statistical testing, the proposed \acrshort{rbf} kernel from \cref{eq:dyngp_kernel} with independent length scales was used due to its simplicity. More complicated kernels were avoided due to challenges with bad local optima during hyperparameter optimization.

The hyperparameters were tuned using the \acrshort{gp} implementation in the popular \textit{scikit-learn} \cite{scikit-learn} Python package. $10$ random restarts were used during optimization to reduce the risk of bad local minima. A lower bound constraint for the length scales was used to avoid obvious overfitting and required the length scales to be greater than $50$. Due to the wide variety of different scenarios on the training set, the hyperparameters are optimized for each simulation iteration. This way, the robustness of the optimization is indirectly tested, as instances of bad local optima are included in the results.


The remaining parameters not found through \acrshort{ml} were tuned through trial and error on a few different trajectories.
The initial state uncertainty is set to $\boldsymbol{P}_0 = 500^2 \cdot \boldsymbol{I}$, which was found to work well during development. The \acrshort{pdaf} parameters are available in \cref{table:stats_pdaf_params} and the \acrshort{sl} parameters is in \cref{table:stats_sl_params}.

\begin{table}[h]
    \centering
    \begin{subtable}{0.49\textwidth}
    \begin{tabular}{|lll|}
        \textit{\textbf{Parameter}} &                  & \textit{\textbf{Value}}     \\ \hline
        Measurement noise           & $\boldsymbol{R}_{\text{PDAF}}$ & $50^2 \cdot \boldsymbol{I}$ \\
        Detection Probability       & $p_D$            & $0.8$                       \\
        Clutter Rate                & $\lambda$        & $2 \cdot 10^{-3}$           \\
        Gate Size                   & $g$              & $2$
    \end{tabular}
    \caption{Parameters used for \acrshort{pdaf} update}
    \label{table:stats_pdaf_params}
    \end{subtable}
    \begin{subtable}{0.49 \textwidth}
        
    \centering
    \begin{tabular}{|lll|}
        \textit{\textbf{Parameter}} &                  & \textit{\textbf{Value}}       \\ \hline
        Noise                       & $\boldsymbol{R}_{\text{SL}}$ & $2000^2 \cdot \boldsymbol{I}$ \\
        Search Radius               & & $500$\\
        & & \\
        & &
    \end{tabular}
    \caption{Parameters used for the SL update}
    \label{table:stats_sl_params}
    \end{subtable}
\end{table}

\subsection{Direct GP}
The excact \acrshort{gp} formulation was implemented using the \texttt{GaussianProcessRegressor} from the popular Python library, \textit{sciki-learn} \cite{scikit-learn}. The library support all kernels introduced in \cref{chap:theory} and supports hyperparameter optimization using multiple restarts to avoid bad local optimas.

The slightly more complicated kernel in \cref{eq:direct_gp_kernel} from \cref{chap:direct_gp} was used. This kernel was preferred over a single \acrshort{rbf} kernel as it simply yielded the best performance across a range of different simulations. The noise term $\sigma^2$ was included as a White kernel and optimized as any other hyperparameter. The parameters was optimized for each test scenario, using $10$ random restarts to reduce the risk of bad local optima. 



%An approximate implementation using \acrshort{svgp} was also attempted without success.The \acrshort{svgp} turned out to be very complicated to make work. Even after several hours of training, the model performed rather poorly, with unrealistic trajectory estimates. It was decided that it was not worth the additional effort of performing statistical testing on this implementation, so only the excact \acrshort{gp} implementation is included.

\section{Results}
The results for straight-line and curved trajectories are available in \cref{table:stats_straight_line_error} and \cref{table:stats_curved_error} respectively which are both found at the end of this chapter. Some hand-picked test scenarios are also included in \cref{chap:gp_ekf_examples} for the different GP-EKF variations.

It is unpractical to compare each different variations of the methods to each other. Instead, this section will take a hierarchical approach, where the GP-EKF using finite-difference for training data is considered the basic configuration of the GP-EKF and will be used as a baseline when comparing the different variants of the GP-EKF. Unless explicitly stated otherwise, it is safe to assume the GP-EKF uses finite-difference and no update step. For comparing specific combinations, the reader is referred to \cref{table:stats_curved_error} and \cref{table:stats_straight_line_error} instead.

As a base of comparison, the GP-EKF and Direct GP are both compared to the \acrshort{cvm} model on both straight-line and curved trajectories in \cref{fig:stats_both_vs_cvm}. On straight-line trajectories, both methods perform worse than the \acrshort{cvm} with higher trajectory error for all quartiles. On curved trajectories, the \acrshort{cvm} struggles as would be expected, whereas both the GP-EKF and Direct GP approaches perform significantly better with both lower median error and spread. The GP-EKF trajectory error has a lower spread than the Direct GP approach, while the Direct GP approach has a lower median error, as can be confirmed in \cref{table:stats_curved_traj_err}. Both the GP-EKF and the Direct \acrshort{gp} approach performs better on curved trajectories than straight-line. After manually inspecting the test cases, it appears that many of the straight-line test cases are slow-moving vessels part of longer curved trajectories.  The training samples used to fit the \acrshort{gp}s in these cases often contain samples from faster moving vessels, making the \acrshort{gp}s overshoot in their predictions. This hypothesis is corroborated by the path error in \cref{table:stats_straight_path_err} as the GP-EKF outperforms the \acrshort{cvm} on straight-line trajectories.

Neither the Direct GP nor the GP-EKF yield consistent uncertainty estimates as found by comparing the \acrshort{nees} to the theoretical $\mathcal{X}^2$ quartiles in \cref{fig:stats_straigth_nees_basic} and \cref{fig:stats_curved_nees_basic}. It is especially bad for straight-line trajectories with estimated quartiles far exceeding the theoretical quartiles, which is likely linked to the increased error for the straight-line trajectories. It is slightly better for curved trajectories, with the $25\%$ and $50\%$ quartiles close to the theoretical values, while the true distribution has a much fatter tail as the estimated $75\%$ quartile is almost twice the theoretical value for both methods.

\subsection{GP-EKF: Finite Difference vs. COG/SOG from AIS}

\begin{figure}[h]
    \centering
    \makebox[\textwidth][c]{
        \begin{subfigure}{0.65\textwidth}
            \includegraphics{figures/straight_line_stats/gp_cog_vs_fd.pdf}
            \caption{Straight-Line Trajectory}
        \end{subfigure}
        \begin{subfigure}{0.65\textwidth}
            \includegraphics{figures/curved_line_stats/gp_cog_vs_fd.pdf}
            \caption{Curved Trajectory}
        \end{subfigure}
    }
    \caption{GP-EKF using finite differences and the \acrshort{cog} and \acrshort{sog} from the AIS dataset on $350$ trajectories. The finite differences approach performs consistently better, with lower median error and spread.}
    \label{fig:stats_gp_ekf_fd_vs_cog}
\end{figure}


A key design choice for the GP-EKF is which data source to use for training. The model can either be trained using the \acrshort{cog} and \acrshort{sog} values contained in the \acrshort{ais} samples or by calculating numerical derivatives of the position through a finite-difference approach. \cref{fig:stats_gp_ekf_fd_vs_cog} compares the difference side-by-side. For straight-line trajectories, the results are remarkably similar. The finite difference approach seems to have a slight advantage on the quartile errors, though the results are well within the margin of error. The finite difference performs slightly better for curved trajectories, especially as time increases, and has both lower median trajectory error and spread. A similar pattern is found for the path error in \cref{table:stats_straight_path_err} and \cref{table:stats_curved_path_err}.

\subsection{GP-EKF: Incorporating positions data}
The effect on the mean prediction for both the \acrshort{pdaf} and the \acrshort{sl} update procedures are compared to the basic GP-EKF configuration in \cref{fig:stats_gp_ekf_with_or_without_update}. The results are remarkably similar for both straight-line and curved trajectories. On closer inspection, there appears to be a slight advantage to using the \acrshort{pdaf} update step, but the improvement is within the margin of error, so it is not possible to draw any conclusion. By looking closely, there also appears to be a slight benefit of using the \acrshort{sl} update, though the improvement is marginal.

However, the \acrshort{pdaf}, and, to some extent, the \acrshort{sl} update has a detrimental effect on the \acrshort{nees}, leading to even more overconfident predictions. This is apparent when comparing the GP-EKF NEES with and without the update steps in \cref{fig:stats_straight_nees_update} and \cref{fig:stats_curved_nees_update}, as the \acrshort{nees} explodes compared to the basic GP-EKF method.










\subsubsection{The modified update}


\begin{figure}
    \centering
    \begin{subfigure}{\textwidth}
        \includegraphics[width=\textwidth]{figures/straight_line_stats/both_vs_cvm.pdf}

        \caption{Straight-line trajectories.}
        \label{fig:stats_curved_both_vs_cvm}
    \end{subfigure}
    \begin{subfigure}{\textwidth}
        \includegraphics[width=\textwidth]{figures/curved_line_stats/both_vs_cvm.pdf}
        \caption{Curved trajectories.}
        \label{fig:stats_straight_both_vs_cvm}
    \end{subfigure}
    \caption{GP-EKF and Direct GP compared to a \acrshort{cvm} on both curved and straigth-line trajectories. In both cases, the results are from 350 simulations.}
    \label{fig:stats_both_vs_cvm}
\end{figure}



\begin{figure}[h]
    \centering
    \begin{subfigure}{1\textwidth}
        \includegraphics[width=\textwidth]{figures/straight_line_stats/gp_vs_update.pdf}
        \caption{Straight-Line Trajectories}
    \end{subfigure}
    \begin{subfigure}{1\textwidth}
        \includegraphics[width=\textwidth]{figures/curved_line_stats/gp_vs_update.pdf}
        \caption{Curved Trajectories}
    \end{subfigure}
    \caption{GP-EKF with and without the \acrshort{sl} and \acrshort{pdaf} updates on curved trajectories for $350$ trajectories. While there are some slight differences, the \acrshort{pdaf} does not appear to have any considerable effect on the trajectory errors. The \acrshort{sl} update have a slight benefit over the basic GP-EKF, with lower spread in the error distribution, especcially for long predition horizons.}
    \label{fig:stats_gp_ekf_with_or_without_update}
\end{figure}

% \begin{figure}[h]
%     \centering
%     \begin{subfigure}{1\textwidth}
%         \includegraphics[width=\textwidth]{figures/straight_line_stats/gp_vs_update_no_P.pdf}
%         \caption{Straight-Line Trajectories}
%     \end{subfigure}
%     \begin{subfigure}{1\textwidth}
%         \includegraphics[width=\textwidth]{figures/curved_line_stats/gp_vs_update_no_P.pdf}
%         \caption{Curved Trajectories}
%     \end{subfigure}
%     \caption{GP-EKF with and without the modified \acrshort{sl} and \acrshort{pdaf} on curved trajectories for $350$ trajectories. The modified \acrshort{sl} performs better than the basic GP-EKF on straight-line trajectories for the upper $75\%$ quartile, while the modifed \acrshort{pdaf} performs about the same as the stock GP-EKF.  The modifed \acrshort{pdaf} performs worse than the basic GP-EKF on curved trajectories, while the modified \acrshort{sl} performs about the same.}
%     \label{fig:stats_gp_ekf_with_or_without_update_no_P}
% \end{figure}

\begin{figure}
    \centering
    \makebox[\textwidth][c]{
        \begin{subfigure}{0.6\textwidth}
            \includegraphics{figures/straight_line_stats/nees_vs_theoretical.pdf}
            \caption{Straight-Line Trajectory NEES Baseline}
            \label{fig:stats_straigth_nees_basic}
        \end{subfigure}
        \begin{subfigure}{0.6\textwidth}
            \includegraphics{figures/curved_line_stats/nees_vs_theoretical.pdf}
            \caption{Curved Trajectory NEES Baseline}
            \label{fig:stats_curved_nees_basic}
        \end{subfigure}
    }
    \makebox[\textwidth][c]{
        \begin{subfigure}{0.6\textwidth}
            \includegraphics{figures/straight_line_stats/nees.pdf}
            \caption{Straight-Line Trajectory with normal update steps}
            \label{fig:stats_straight_nees_update}
        \end{subfigure}
        \begin{subfigure}{0.6\textwidth}
            \includegraphics{figures/curved_line_stats/nees.pdf}
            \caption{Curved Trajectory with normal update steps}
            \label{fig:stats_curved_nees_update}
        \end{subfigure}
    }

    %\makebox[\textwidth][c]{
    %    \begin{subfigure}{0.6\textwidth}
    %        \includegraphics{figures/straight_line_stats/nees_no_P.pdf}
    %        \caption{Straight-Line Trajectory with modified update steps}
    %    \end{subfigure}
    %    \begin{subfigure}{0.6\textwidth}
    %        \includegraphics{figures/curved_line_stats/nees_no_P.pdf}
    %        \caption{Curved Trajectory with modified update steps}
    %    \end{subfigure}
    %}
    \caption{The NEES compared between the different methods. The top row compares the \acrshort{nees} for the Direct GP and the base GP-EKF to the theoretical quartile values for the $\mathcal{X}^2$ distribution (black lines), and is intended as a base of comparison for the subsequent rows. The second row compares the effect of the normal update steps for the GP-EKF.}

\end{figure}

% On straight-line trajectories, the $25\%$ and $50\%$ quartiles correspond well with the theoretical values for the $\mathcal{X}^2$ distribution for all methods. The \acrshort{pdaf} update step does, however, reduce the predicted uncertainty, leading to increased overconfidence. In addition, the estimated \acrshort{nees} distribution has a much fatter right-tail than the theoretical $\mathcal{X}^2$ distribution, leading to an estimated $75\%$ quartile far outside the theoretical values.

% On curved trajectories, the GP-EKF with and without PDAF is consistently overconfident. The theoretical quartiles are not plotted due to the large y-axis. The direct \acrshort{gp} approach performs significantly better, which is why a separate plot in \cref{fig:stats_curved_nees_direct} is added to compare it to theoretical quartiles.




\begin{table}[b]
    \begin{subtable}{\textwidth}
        \makebox[\textwidth][c]{
            \begin{tabular}{lllrrrrr}
                \toprule
                        &                & Time [Minutes]        & 5       & 10      & 15       & 20       & 25       \\
                Summary & Method         & Training Source       &         &         &          &          &          \\
                \midrule
                Mean    & CVM            & COG/SOG from AIS      & 183     & 405     & 652      & 802      & 1151     \\
                        & Direct GP      & \bf Position          & \bf 532 & \bf 685 & \bf 915  & \bf 1274 & \bf 1636 \\
                        & GP-EKF         & COG/SOG from AIS      & 350     & 704     & 1153     & 1704     & 1975     \\
                        &                & \bf Finite Difference & \bf 359 & \bf 681 & \bf 1060 & \bf 1522 & \bf 1909 \\
                        & GP-EKF w/ PDAF & COG/SOG from AIS      & 341     & 679     & 1030     & 1434     & 1638     \\
                        &                & \bf Finite Difference & \bf 341 & \bf 638 & \bf 972  & \bf 1396 & \bf 1753 \\
                        & GP-EKF w/ SL   & COG/SOG from AIS      & 341     & 680     & 1105     & 1634     & 1879     \\
                        &                & \bf Finite Difference & \bf 353 & \bf 667 & \bf 1029 & \bf 1476 & \bf 1778 \\
                \midrule
                Median  & CVM            & COG/SOG from AIS      & 96      & 210     & 358      & 464      & 721      \\
                        & Direct GP      & \bf Position          & \bf 312 & \bf 503 & \bf 649  & \bf 992  & \bf 1266 \\
                        & GP-EKF         & COG/SOG from AIS      & 292     & 577     & 994      & 1598     & 1876     \\
                        &                & \bf Finite Difference & \bf 297 & \bf 538 & \bf 835  & \bf 1285 & \bf 1474 \\
                        & GP-EKF w/ PDAF & COG/SOG from AIS      & 279     & 550     & 807      & 1141     & 1263     \\
                        &                & \bf Finite Difference & \bf 289 & \bf 511 & \bf 752  & \bf 1115 & \bf 1563 \\
                        & GP-EKF w/ SL   & COG/SOG from AIS      & 285     & 545     & 939      & 1530     & 1869     \\
                        &                & \bf Finite Difference & \bf 287 & \bf 518 & \bf 813  & \bf 1246 & \bf 1346 \\
                \bottomrule
            \end{tabular}
        }
        \caption{Trajectory errors in meters}
        \label{table:stats_straight_traj_err}
        \vspace*{0.5cm}
    \end{subtable}
    \begin{subtable}{\textwidth}
        \makebox[\textwidth][c]{
            \begin{tabular}{lllrrrrr}
                \toprule
                        &                & Time [Minutes]        & 5       & 10      & 15      & 20      & 25      \\
                Summary & Method         & Training Source       &         &         &         &         &         \\
                \midrule
                Mean    & CVM            & COG/SOG from AIS      & 52      & 150     & 284     & 344     & 582     \\
                        & Direct GP      & \bf Position          & \bf 258 & \bf 239 & \bf 344 & \bf 473 & \bf 701 \\
                        & GP-EKF         & COG/SOG from AIS      & 78      & 140     & 202     & 249     & 286     \\
                        &                & \bf Finite Difference & \bf 80  & \bf 148 & \bf 224 & \bf 265 & \bf 332 \\
                        & GP-EKF w/ PDAF & COG/SOG from AIS      & 100     & 198     & 311     & 354     & 389     \\
                        &                & \bf Finite Difference & \bf 81  & \bf 149 & \bf 234 & \bf 279 & \bf 404 \\
                        & GP-EKF w/ SL   & COG/SOG from AIS      & 76      & 132     & 188     & 231     & 266     \\
                        &                & \bf Finite Difference & \bf 79  & \bf 140 & \bf 207 & \bf 243 & \bf 306 \\
                \midrule
                Median  & CVM            & COG/SOG from AIS      & 28      & 70      & 136     & 223     & 411     \\
                        & Direct GP      & \bf Position          & \bf 76  & \bf 120 & \bf 197 & \bf 258 & \bf 419 \\
                        & GP-EKF         & COG/SOG from AIS      & 58      & 99      & 142     & 180     & 233     \\
                        &                & \bf Finite Difference & \bf 65  & \bf 105 & \bf 157 & \bf 167 & \bf 223 \\
                        & GP-EKF w/ PDAF & COG/SOG from AIS      & 60      & 104     & 149     & 184     & 240     \\
                        &                & \bf Finite Difference & \bf 66  & \bf 104 & \bf 147 & \bf 179 & \bf 205 \\
                        & GP-EKF w/ SL   & COG/SOG from AIS      & 55      & 94      & 132     & 159     & 192     \\
                        &                & \bf Finite Difference & \bf 63  & \bf 101 & \bf 139 & \bf 155 & \bf 199 \\
                \bottomrule
            \end{tabular}
        }
        \caption{Path error in meters}
        \label{table:stats_straight_path_err}
    \end{subtable}
    \caption{Error summary for $350$ straight-line trajectories. Mean and median summary statistics are calculated for the trajectory and path error at fixed timestamps. Linear interpolation is used between samples. Errors for short trajectories are not extrapolated, and therefore not included in the $20$ and $25$ minute bins.}
    \label{table:stats_straight_line_error}
\end{table}

\begin{table}[b]
    \begin{subtable}{\textwidth}
        \makebox[\textwidth][c]{
            \begin{tabular}{lllrrrrr}
                \toprule
                        &                & Time [Minutes]         & 5       & 10      & 15      & 20       & 25       \\
                Summary & Method         & Training Source        &         &         &         &          &          \\
                \midrule
                Mean    & CVM            & COG/SOG from AIS       & 440     & 1071    & 1898    & 2425     & 3313     \\
                        & Direct GP      & \bf Position           & \bf 366 & \bf 522 & \bf 771 & \bf 993  & \bf 1298 \\
                        & GP-EKF         & COG/SOG from AIS       & 352     & 674     & 950     & 1327     & 1856     \\
                        &                & \bf Finite Difference  & \bf 329 & \bf 575 & \bf 823 & \bf 1081 & \bf 1388 \\
                        & GP-EKF w/ PDAF & COG/SOG from AIS       & 478     & 904     & 1255    & 1473     & 1824     \\
                        &                & \bf Finite Difference  & \bf 371 & \bf 610 & \bf 835 & \bf 1056 & \bf 1220 \\
                        & GP-EKF w/ SL   & COG/SOG from AIS       & 340     & 636     & 874     & 1194     & 1592     \\
                        &                & \bf Finite Difference  & \bf 324 & \bf 555 & \bf 795 & \bf 1030 & \bf 1312 \\
                \hline
                Median  & CVM            & COG/SOG from AIS       & 167     & 452     & 1009    & 1561     & 2359     \\
                        & Direct GP      & \bf Position           & \bf 229 & \bf 364 & \bf 550 & \bf 705  & \bf 942  \\
                        & GP-EKF         & COG/SOG from AIS       & 257     & 483     & 724     & 1078     & 1505     \\
                        &                & \bf Finite Difference  & \bf 255 & \bf 463 & \bf 635 & \bf 844  & \bf 1120 \\
                        & GP-EKF w/ PDAF & COG/SOG from AIS       & 368     & 654     & 923     & 1128     & 1327     \\
                        &                & \bf Finite Difference  & \bf 315 & \bf 493 & \bf 651 & \bf 704  & \bf 854  \\
                        & GP-EKF w/ SL   & COG/SOG from AIS       & 251     & 452     & 646     & 989      & 1357     \\
                        &                & \bf  Finite Difference & \bf 256 & \bf 434 & \bf 626 & \bf 789  & \bf 1113 \\
                \bottomrule
            \end{tabular}
        }
        \caption{Trajectory Error in meters}
        \label{table:stats_curved_traj_err}
        \vspace*{0.5cm}
    \end{subtable}
    \begin{subtable}{\textwidth}
        \makebox[\textwidth][c]{
            \begin{tabular}{lllrrrrr}
                \toprule
                        &                & Time [Minutes]         & 5        & 10       & 15      & 20      & 25       \\
                Summary & Method         & Training Source        &          &          &         &         &          \\
                \midrule
                Mean    & CVM            & COG/SOG from AIS       & 204      & 574      & 1072    & 1518    & 2045     \\
                        & Direct GP      & \bf Position           & \bf 175  & \bf 231  & \bf 403 & \bf 596 & \bf 827  \\
                        & GP-EKF         & COG/SOG from AIS       & 143      & 269      & 356     & 484     & 687      \\
                        &                & \bf Finite Difference  & \bf 160  & \bf 291  & \bf 412 & \bf 466 & \bf 509  \\
                        & GP-EKF w/ PDAF & COG/SOG from AIS       & 177      & 355      & 587     & 775     & 1218     \\
                        &                & \bf Finite Difference  & \bf 155  & \bf 277  & \bf 413 & \bf 515 & \bf  655 \\
                        & GP-EKF w/ SL   & COG/SOG from AIS       & 137      & 240      & 300     & 392     & 522      \\
                        &                & \bf Finite Difference  & \bf 151  & \bf 265  & \bf 375 & \bf 437 & \bf 523  \\
                \hline
                Median  & CVM            & COG/SOG from AIS       & 54       & 265      & 715     & 1138    & 1863     \\
                        & Direct GP      & \bf Position           & \bf 81   & \bf 140  & \bf 234 & \bf 336 & \bf 447  \\
                        & GP-EKF         & COG/SOG from AIS       & 99       & 152      & 216     & 295     & 408      \\
                        &                & \bf Finite Difference  & \bf 106  & \bf 180  & \bf 323 & \bf 336 & \bf  377 \\
                        & GP-EKF w/ PDAF & COG/SOG from AIS       & 97       & 157      & 290     & 401     & 908      \\
                        &                & \bf  Finite Difference & \bf  101 & \bf  182 & \bf 304 & \bf 371 & \bf 402  \\
                        & GP-EKF w/ SL   & COG/SOG from AIS       & 92       & 133      & 195     & 227     & 302      \\
                        &                & \bf  Finite Difference & \bf 101  & \bf 173  & \bf 290 & \bf 309 & \bf 395  \\
                \bottomrule
            \end{tabular}
        }
        \caption{Path error in meters}
        \label{table:stats_curved_path_err}
    \end{subtable}
    \caption{Error summary for $350$ curved trajectories. Mean and median summary statistics are calculated for the trajectory and path error at fixed timestamps. Linear interpolation is used between samples. Errors for short trajectories are not extrapolated, and therefore not included in the $20$ and $25$ minute bins.}
    \label{table:stats_curved_error}
\end{table}