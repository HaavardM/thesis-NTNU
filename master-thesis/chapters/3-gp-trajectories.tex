\chapter{Non-parametric trajectories using Gaussian Process}\label{chap:impl}

\section{Learning trajectories directly from data}
Perhaps the most direct approach is to use a \acrshort{gp} to model a function $f: \mathcal{R}^5 \to \mathcal{R}^2$, mapping a ships current position $\boldsymbol{p}_t$, \acrshort{cog} $\psi_t$ and \acrshort{sog} $v_t$ to a position at time $t+\tau$. The method is formulated in mathematical terms in \cref{eq:gp_direct}. The model can then be queried about likely future positions by evaluating $\boldsymbol{f}$ with the desired initial conditions and time horizon, and then condition the \acrshort{gp} on close trajectories. 
The method simply boils down to using the \acrshort{gp} kernels to find trajectories with similar initial conditions, and use those to predict future positions.

\begin{subequations}\label{eq:gp_direct}
\begin{align}
    \boldsymbol{f}(\boldsymbol{x}) &= \boldsymbol{p}_{t+\tau} \label{eq:gp_direct_f}, \quad \boldsymbol{x} = \begin{bmatrix} \boldsymbol{p}_t & \psi_t & v_t & \tau\end{bmatrix}\\
    \boldsymbol{f}(\boldsymbol{x}) &\sim \text{GP}(\boldsymbol{m}(\boldsymbol{x}), K(\boldsymbol{x}, \boldsymbol{x}))\label{eq:gp_direct_f_dist}
\end{align} 
\end{subequations}

The benefits of this approach include:
\begin{description}
    \item[Continuous formulation] The model directly models the position at any time $t+\tau$ and do not require discretization. 
    \item[Simple Problem formulation] The model requires few components.
    \item[Easy incorporation of available AIS data] The reported \acrshort{cog} and \acrshort{sog} can easily be incorporated into the similarity measure, utilizing more of the available data.
\end{description}

However, the model is exclusively data-driven and makes it more challenging to incorporate model information. The high number of input variables also makes it more challenging to interpret how the model makes prediction. This makes tuning more challenging. This model further assumes that the trajectory is jointly Gaussian, making it more challenging to describe branching trajectories.

\section{Non-parametric dynamic system using AIS tracking data}
On of the major issues with the direct approach is the unimodal assumption of using a \acrshort{gp}. This works well as long as vessels tends to agree on a specific trjaectory, but fails as soons there are multiple branching trajectories.

In \cite{pedestrian}, a \acrshort{gp} was used to model the trajectory patterns of pedestrians tracked using computer vision. Rather than directly describing trajectories, the paper proposed to simulate trajectories from a non-parametric dynamical model $\Delta\boldsymbol{x}_{t+1} = \vec{f}(\boldsymbol{x}_t)$ where the increments are expressed using a \acrshort{gp}. The trajectories were then simulated using two different approaches: 
\begin{enumerate}
    \item By assuming $p(\boldsymbol{x})$ is always unimodal and Gaussian, the GP-EKF introduced in \cite{gpekf} was used to simulate the trajectory for multiple timesteps, using the dynamical \acrshort{gp} model as both prediction and measurement model. This formulation is unable to express multimodal uncertainty.
    \item In order to retain the inherent multimodaility, a sequential Monte-Carlo approach (i.e. the prediction step of a particle filter) was used to keep track of multiple modes (i.e. branching trajectories), at the cost of computational complexity. 
\end{enumerate}

In this section, a similar method is proposed for long-term vessel prediction. The vessel trajectory $\boldsymbol{\mathcal{T}}$ can be expressed using the dynamical system
\begin{subequations}
\begin{align}
    \boldsymbol{x}_{t+1} &= \boldsymbol{x}_t + \vec{f}(\boldsymbol{x}_t)\\
    \mathcal{T}_t &= \boldsymbol{x}_t + \epsilon, \quad \epsilon \sim \mathcal{N}(0, \sigma^2)
\end{align}
\end{subequations}
The function $\vec{f}(\cdot): \mathcal{R}^2 \to \mathcal{R}^2$ denotes the vector field describing virtual forces acting on a vessel. Normally this function is parametrized explicitly and the solution is solved either analytically or numerically depending on the complexity of the model. However, in the case of long-term prediction, the dynamics $\vec{f}(\cdot)$ is unknown and is unlikely stationary. Instead, the goal of this chapter is to use a \acrshort{gp} to create a non-parametric representation of the dynamics $\vec{f}(\cdot)$ by learning from historical trajectories of other vessels.

\begin{equation}\label{eq:gp_vec_field}
    \vec{f}(\boldsymbol{x}) = \begin{bmatrix} f_x (\boldsymbol{x})\\ f_y (\boldsymbol{x})\end{bmatrix} \sim \text{GP} \big(\begin{bmatrix} m_x(\boldsymbol{x})\\m_y(\boldsymbol{x})\end{bmatrix}, \ \begin{bmatrix}
    K_{xx}(\boldsymbol{x}, \boldsymbol{x}') & K_{xy}(\boldsymbol{x}, \boldsymbol{x}') \\ K_{xy}(\boldsymbol{x}, \boldsymbol{x}')^\intercal & K_{yy}(\boldsymbol{x}, \boldsymbol{x}')
    \end{bmatrix}\big) 
\end{equation}

The benefits of this formulation include:
\begin{description}
    \item[Easy incorporation of existing data] The model can easily be trained on partial data. Only the gradients of any historical trajectories are really needed.
    \item[Few constraining assumptions] The dynamical model is not constrained by any specific parametrization, while still allows prior knowledge such as smoothness to be incorporated into the model.
    \item[Branching trajectories] The dynamical formulation only assume unimodal Gaussian increments, while the full trajectory can still be multimodal. Though not analytically tractable, the multimodal trajectory can be found using sampling based methods. 
\end{description}

The major downside is however that the model only learns gradients from data and relies heavily on numerical simulations in order to get the desired trajectories.

The model could further be improved by clustering the samples based on similar behaviour for different ships, such as similarities in velocity. Separate \acrshort{gp}s could then be trained on the different subsets of data.

\cite{heinonen2018learningode} use a similar formulation to infer arbitrary, non-linear ODE models from sparse data and predict the dynamics far into the future. 


\subsection{Choice of kernels}
The kernels is in this case used as a similarity measure between two trajectories. This boils down to fin




\section{Branching Trajectories}


\subsection{Mixture of Gaussian Process Experts}

