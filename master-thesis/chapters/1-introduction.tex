\chapter{Introduction}
% Something about the need for long-term prediction of vessel trajectories
% Something about Big Data AIS - the availability of data
% Something about how we define long-term trajectory prediction
\section{The Need For Quantifying Uncertianty}
Predicting the trajectory of a vessel based on the limited information available is no easy task. Even given historical trajectories of the same vessel, there is no way to guarantee that the vessel will behave identically in the future. In the end, a vessel can take an indefinite amount of different trajectories to reach a given destination. 

The ability of a model to express uncertainty is therefore crucial. Ideally, the model should make accurate predictions while also quantify the wide range of different trajectories a vessel might follow. 

\subsection{Gaussian Processes}
% High level introduction to the Gaussian process and why it may be a natural fit for this problem
\section{Previous Work}
There have been several attempts on utilizing Big Data \acrshort{ais} to improve long-term vessel trajectory prediction. 

\subsection{Clustering-based method}
These methods typically work by clustering historical \acrshort{ais} data and then create representable paths to use when predicting vessels' future trajectories. 


Traditional clustering methods, such as k-means or DBSCAN, tend to focus on clustering point values. In the context of trajectory prediction, the trajectories would then be clustered as a whole. The trajectory clustering algorithm TRACLUS was therefore introduced by \cite{traclus} where it was applied to hurricane trajectory and animal movement data. The key observation was that trajectories might have portions that share common behavior, while the entire trajectory might still differ. TRACLUS allows clustering trajectories based on common sub-trajectories and works by partitioning the trajectories into smaller line segments and then groups similar segments into clusters.  







%.... somnething about why these previous fails

\section{Thesis Structure}
The Gaussian Process will be presented in the first chapter. Inspired by the excellent introduction in \cite{rasmussen}, the reasoning, intuition, and mathematics behind the Gaussian Process will be explored to lay the necessary foundation for the rest of this thesis. A basic understanding of calculus, statistics, and probability theory is assumed, but all necessary subjects are otherwise introduced.  

Once the foundation for the Gaussian Processes has been covered, the next chapters will move on to the different ways they can be applied to the long-term prediction problem. In the following chapter, the first approach is introduced, where a vessel's trajectory is described as a function of time and modeled directly as a Gaussian Process. This method is similar to interpolating and averaging similar trajectories, though with the added benefit of quantifying the uncertainty.

The following chapter then continues with a more indirect approach. The Gaussian Process is used as part of a dynamical system and then simulated iteratively to get the trajectories. This method is inspired by the work in \cite{pedestrian,gpekf}, and can be seen as an extension of the Kalman filter. Familiarity with sensor-fusion methods is beneficial but not required to follow the derivations.

Once the methods are introduced in their respective chapter, statistical analysis of the methods will be performed. Both methods are applied on a real \acrshort{ais} dataset, and the results are compared to a simple, constant velocity model.  We will consider both the empirical results as well the more theoretical strengths and weaknesses. 

Finally, we will summarize our findings, as well as discuss possible extensions to this work.



