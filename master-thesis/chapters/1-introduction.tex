\chapter{Introduction}
Humans are lazy and have always strived to avoid repeating tasks. Time is better spent doing a more rewarding task, leaving machines to do tasks that are not worth spending human resources. Automation has typically been chiefly focused on industrial applications where there is a direct economic incentive to automate tasks. The processes are also isolated and highly repeatable, which avoids the need for human interaction as much as possible. While the idea of replacing humans in more complicated situations outside of the factory floor is not new, it is only in the last few decades that machines have become sophisticated enough actually to replace their human alternatives in more everyday tasks. Outside of the constrained environment of the factory floor, machines need to be able to understand and adapt to the wide range of different scenarios which may occur in the chaotic environment designed for and by humans. 

This thesis will focus on one such automation task, namely \textit{\acrfull{asv}}. Humans are prone to loss of focus, tiredness, and limited attention span. Combined with large and powerful machines, they are a potential danger to both themselves and others. It is estimated that over $75\%$ of maritime accidents are attributed to human errors \cite{Tengesdal2020RiskbasedAM}, motivating the need for automated alternatives to human captains. While altogether avoiding humans at sea would significantly reduce accidents, it is simply unrealistic to achieve. Hence, autonomous vessels need to learn to understand and cooperate with other human-driven vessels. However, it is easy to take humans' remarkable ability to understand patterns for granted. While people certainly make mistakes, they can understand highly complex scenarios and infer likely outcomes given their prior experiences in similar situations. An \acrshort{asv} will need similar abilities in order to operate without putting humans at risk in a reliable way. A key aspect to solving this problem is a robust \textit{\acrfull{colav}} system, which can both reactively and proactively avoid collisions. In order to do so, the \acrshort{colav} need accurate situational awareness, which includes both information about the current situation as well as how the future might unfold in order to avoid high-risk situations proactively. This thesis will focus on the latter, namely how to predict the future behavior of nearby vessels.

 Knowing the future behavior of surrounding vessels, even only probabilistically, allows the \acrshort{colav} system to avoid scenarios with increased risk of collision actively. A simple solution is to assume near a constant speed and course of any vessels of interest and is a commonly used assumption for obstacle models in \acrshort{colav} research \cite{kuwata, }. While certainly a reasonable solution on the open sea where vessels spend a significant portion of the time in straight-line trajectories, it quickly fails when vessels navigate constrained areas such as fjords or close to shore where more complicated maneuvering is necessary. Considering the usual prediction horizon of $5-15$ minutes into the future for the typical \acrshort{colav} application \cite{dalsnes}, assuming near a constant speed and course is not sufficient. 
 However, the vessels still tend to follow typical traffic lanes and behave somewhat predictably. Combined with modern machine learning, the question becomes how this data can be utilized to improve the prediction of future trajectories, and thereby allowing \acrshort{asv}s to avoid high-risk \acrshort{colav} scenarios proactively. Ideally, the \acrshort{asv} should learn from the available data and use it to recognize patterns to infer likely outcomes.

% Something about Big Data AIS - the availability of data
% Something about how we define long-term trajectory prediction
\section{The Need For Quantifying Uncertianty}
Predicting the trajectory of a vessel based on the limited information available is no easy task. Even given historical trajectories of the same vessel, there is no way to guarantee that the vessel will behave identically in the future. In the end, a vessel can take an indefinite amount of different trajectories to reach a given destination. 

The ability of a model to express uncertainty is therefore crucial. Ideally, the model should make accurate predictions while also quantify the wide range of different trajectories a vessel might follow. The need for quantifying uncertainty lends itself nicely to a Bayesian approach, as the uncertainty becomes a first-class citizen in the model. Such an approach allows the \acrshort{colav} to express \textit{beliefs} about the possible outcomes and does not hide the stochastic nature of the problem it is trying to solve. 

\subsection{Gaussian Processes}
The \textit{\acrfull{gp}} is a stochastic process based on the multivariate Gaussian Distribution. It is an interpolation method with use cases in a wide range of fields, such as environmental science \cite{kriging}, medicine \cite{medical_gp,medicine_gp_2}, cognitive research \cite{gp_cognitive} and optimization \cite{brochu2010tutorial} to name a few. It was first used by Danie G. Krige in his master thesis \cite{krige1951statistical} where he used it to find the most likely spatial distribution of gold based on samples from only a few boreholes. \acrshort{gp} are therefore sometimes referred to as \textit{kriging} in the literature, especially for geospatial statistics or related fields. 
The \acrshort{gp} is a non-parametric approach, and it is well suited for approximating black-box systems. While comparable to other interpolation methods such as splines, \acrshort{gp}s has the added benefit of expressing uncertainty based on the available data. Therefore, it is commonly interpreted as a statistical distribution over functions. Prior beliefs can be conditioned on available data to form posterior beliefs about the true underlying function $f(x)$. This Bayesian interpretation, in combination with its flexibility as a non-parametric approach, makes \acrshort{gp} a very powerful tool. \acrshort{gp}s will be covered in great detail in \cref{chap:theory}. 

The functional interpretation appears to be a good fit for the long-term prediction problem. By considering a vessel's trajectory as a function of time $f(t)$, then the \acrshort{gp} framework should, in theory, be able to express the vessel's trajectory, including the inherent uncertainty of predicting future trajectories. In practice, however, the computational complexity of \acrshort{gp}s may limit the usability of \acrshort{gp}s.  

This raises a few research questions which this thesis will attempt to answer:
\begin{enumerate}
	\item How can \acrshort{gp}s be used to model the long-term vessel trajectory of new vessels?   
	\item Is \acrshort{gp}s computationally feasible to use in the context of \acrshort{ais} Big Data?
	\item Is a \acrshort{gp} able to provide consistent uncertainty, which reflects the true error rate?
\end{enumerate}

Especially question 3 is a crucial focus in this thesis. If the \acrshort{gp} cannot provide consistent uncertainty estimates, it then offers little additional value over more straightforward data-driven approaches. 


This thesis will propose two alternative solutions to the long-term trajectory prediction problem, both based on the \acrshort{gp} framework. The first method will attempt to model the trajectory directly as a function of time and use a \acrshort{gp} to express this function. The other solution is an indirect approach attempting to learn an unknown motion model and then simulate the trajectory numerically. Both methods will be rigorously tested on a real-world \acrshort{ais} dataset.

\section{Thesis Structure}
The thesis will first review existing work on applying the \acrshort{gp} framework for trajectory prediction in \cref{chap:prior_work}. As there is little work available on using \acrshort{gp}s with \acrshort{ais} data, solutions from related fields as well as some approaches not based on \acrshort{gp}s will also be presented as inspiration. The thesis then moves on to \cref{chap:ais} which offers an introduction to \acrshort{ais} and the dataset used in this thesis. Any preprocessing and filtering of the dataset are included in this chapter.

The Gaussian Process will then be introduced in greater detail in \cref{chap:theory}. The chapter is inspired by the excellent introduction from \cite{rasmussen}, and the reasoning, intuition, and mathematics behind the Gaussian Process will be explored to lay the necessary foundation for the rest of this thesis. A basic understanding of calculus, statistics, and probability theory is assumed, but all necessary subjects are otherwise introduced.  

Once the foundation for the Gaussian Processes have been covered, \cref{chap:direct_gp} will apply the \acrshort{gp} framework to model a vessel's trajectory directly as a function of time. This method will be referred to as the \textit{direct \acrshort{gp} approach} throughout the rest of this thesis.
\cref{chap:gp_ekf} then proposes a more indirect approach, using the \acrshort{gp} framework to model an unknown differential equation describing the trajectory gradients. The trajectories are simulated numerically using a \acrshort{ekf} prediction scheme. This method is inspired by the work in \cite{pedestrian,gpekf}, and can be seen as an extension of the Kalman filter. Again, familiarity with sensor-fusion methods is beneficial but not required to follow the derivations.

Once the methods are introduced in their respective chapter, the thesis will move on to statistical testing in \cref{chap:stat_testing}. This chapter will cover the implementation details and present the results from rigorous statistical testing to see how the methods perform on real-world \acrshort{ais} data. The results are compared to a simple \acrshort{cvm} model as a base of comparison. 

\cref{chap:discussion} will then discuss the results from statistical testing, as well as address theoretical concerns regarding the proposed methods. 
Finally, \cref{chap:conclusion} will summarize the findings in conclusion before finishing with possible extensions to this work.



