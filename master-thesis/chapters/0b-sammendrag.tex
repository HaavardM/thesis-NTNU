\chapter*{Sammendrag}
For at autonome overflate-fartøyer (ASVer) skal kunne operere trygt er det essensielt med robuste antikollisjonssystemer. Slike systemer innebærer ikke bare at et fartøy må kunne reagere i det det oppstår farlige situasjoner, men også evnen til å proaktivt unngå situasjoner med høy risiko. Fartøyene er dermed nødt til å gjenkjenne ulike scenarioer og kunne planlegge for potensielle hendelser frem i tid. Denne fremtidsforståelsen er temaet i denne oppgaven og målet er å utforske hvordan historisk data fra Automatisk Identifikasjonssystem (AIS) kan brukes til å predikere skips fremtidige bevegelser.

Mer spesifikt foreslår denne oppgaven bruk av Gaussiske Prosesser til å lære bevegelsesmønsteret til skip i ulike scenarier basert på historiske data. Motivasjonen bak Gaussiske Prosesser er tolkningen som en statistisk fordeling over unkjente funksjoner, og en slik representasjon kan dermed naturlig inkorperere usikkerhetsestimater. I tråd med Gaussiske Prosesser blir et Bayesiansk statistisk rammeverk anvendt for å ta hensyn til usikkerhet.

Oppgaven foreslår to forskjellige måter å anvende Gaussiske Prosesser for prediksjon av skips bevegelser. Begge metodene blir testet med data fra et reelt datasett hentet fra Trondheimsfjorden i løpet av ett år, og en større statistisk sammenligning av metodene gjennomføres for å se hvordan metodene presterer. For å undersøke om metodene er i stand til å utrykke en usikkerhet som stemmer overens med den sanne feilraten, gjennomføres en konsistensanalyse.

Den første metoden modellerer posisjonen direkte som en funksjon av tid og fungerer bra så fremt det ikke er forgreninger i sannsynlige baner og banen kan utrykkes unimodalt. Dette er den desidert enkleste formuleringen, men den har noen teoretiske begrensninger.
Den andre metoden fokuserer derimot på å modellere en ukjent bevegelsesmodell og anvender en Gaussisk Prosess til å beskrive hastighetsvektoren som et vektorfelt. En måte for å simulere baner blir så foreslått, med ekstra fokus på å propagere usikkerhet gjennom simuleringen. Metoden fungerer bra, men er sensitiv til valg av parametre og viser seg å være mindre robust i praksis. 




For at autonome overflate-fartøyer (ASVer) skal kunne operere trygt er det essensielt med robuste antikollisjonssystemer. Slike systemer innebærer ikke bare at et fartøy må kunne reagere i det det oppstår farlige situasjoner, men også evnen til å proaktivt unngå situasjoner med høy risiko. Fartøyene er dermed nødt til å gjenkjenne ulike scenarioer og kunne planlegge for potensielle hendelser frem i tid. Denne fremtidsforståelsen er temaet i denne oppgaven, og målet er å utforske hvordan historisk data fra Automatisk Identifikasjonssystem (AIS) kan brukes til å predikere skips fremtidige bevegelser.

Oppgaven foreslår to metoder basert på et rammeverk for en Gaussisk Prosess (GP). Gaussiske Prosessers intuitive tolkning som en statistisk fordeling over funksjoner gjør at prediksjonene også kan innlemme usikkerhet som en sentral del av modellen. Et Bayesiansk statistisk rammeverk brukes til å alltid eksplisitt vurdere den underliggende usikkerheten når prediksjoner utføres.

Den første foreslåtte metoden bruker et rammeverk basert på Gaussiske Prosesser direkte for å modellere posisjon i banen som en funksjon av tid. Denne tilnærmingen fungerer rimelig bra, bortsett fra i nærvær av forgrenede trafikkfelt. Formuleringen av metoden gjør strenge antakelser om unimodalitet og kan ikke representere noen form for multimodal usikkerhet.

Som en mer indirekte tilnærming forsøker den andre metoden å bruke en Gaussisk Prosess til å beskrive en latent bevegelsesmodell og bruke den til å simulere baner numerisk. Denne formuleringen er langt mer fleksibel og er i teorien i stand til å uttrykke multimodale fordelinger for de predikerte banene. Å kombinere denne tilnærmingen med et prediksjonssystem basert på et Utvidet Kalman Filter (EKF) for å simulere baner fungerer bra så lenge banene er tilstrekkelig glatte, slik at en Taylor-tilnærming av bevegelsesmodellen fungerer som en rimelig tilnærming. Disse antagelsene gjør imidlertid denne metoden mer skjør enn den første metoden.

Begge metodene testes grundig på et reelt AIS-datasett samlet fra Trondheimsfjorden i løpet av ett år, og den statistiske ytelsen til begge metodene sammenlignes. Konsistensen av usikkerhetsestimatene blir også testet for å undersøke om metodene er i stand til å nøyaktig kunne representere den underliggende usikkerheten.





