\chapter*{Sammendrag}
For at autonome overflate-fartøyer (ASVer) skal kunne operere trygt er det essensielt med robuste antikollisjonssystemer. Slike systemer innebærer ikke bare å reagere i det det oppstår farlige situasjoner, men også å proaktivt unngå situasjoner med høy risiko. De er dermed nødt til å gjenkjenne ulike scenarioer og kunne planlegge for potensielle hendelser frem i tid. Denne fremtidsforståelsen er temaet i denne oppgaven og målet er å utforske hvordan historisk data fra Automatisk Identifikasjonssystem (AIS) kan brukes til å predikere skips fremtidige bevegelser.

Mer spesifikt foreslår denne oppgaven bruk av Gaussiske Prosesser til å lære bevegelsesmønsteret til skip i ulike scenarioer fra historiske data. Motivasjonen bak Gaussiske Prosesser er tolkningen som en statistisk fordeling over unkjente funksjoner, og en slik representasjon kan dermed naturlig inkorperere usikkerhetsestimater. I tråd med Gaussiske Prosesser blir et Bayesiansk statistisk rammeverk anvendt for å ta hensyn til usikkerhet.

Oppgaven foreslår to forskjellige måter å anvende Gaussiske Prosesser til prediksjon av skips bevegelser. Begge metodene blir testet med data fra et reelt datasett hentet fra Trondheimsfjorden i løpet av ett år, og en større statistisk sammenligning av metodene blir så gjennomført for å se hvordan metodene fungerer. Konsistensanalyse blir utført for å undersøke om metodene er i stand til å utrykke en usikkerhet som stemmer overens med den sanne feilraten.  
Den første metoden modellerer posisjonen direkte som en funksjon av tid og fungerer bra så fremt det ikke er forgreninger i sannsynlige baner og banen kan utrykkes unimodalt. Dette er den desidert enkleste formuleringen, men den har noen teoretiske begrensninger.
Den andre metoden fokuserer derimot på å modellere en ukjent bevegelsesmodell og anvender en Gaussisk Prosess til å beskrive hastighetsvektoren som et vektorfelt. En måte for å simulere baner blir så foreslått, med ekstra fokus på å propagere usikkerhet gjennom simuleringen. Metoden fungerer bra, men er sensitiv til valg av parametre og viser seg å være mindre robust i praksis. 







