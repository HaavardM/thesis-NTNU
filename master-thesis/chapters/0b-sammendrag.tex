\chapter*{Sammendrag}
For at autonome overflate-fartøyer (ASVer) skal kunne operere trygt er det essensielt med robuste antikollisjonssystemer. Slike systemer innebærer ikke bare at et fartøy må kunne reagere i det det oppstår farlige situasjoner, men også evnen til å proaktivt unngå situasjoner med høy risiko. Fartøyene er dermed nødt til å gjenkjenne ulike scenarioer og kunne planlegge for potensielle hendelser frem i tid. Denne fremtidsforståelsen er temaet i denne oppgaven, og målet er å utforske hvordan historisk data fra Automatisk Identifikasjonssystem (AIS) kan brukes til å predikere skips fremtidige bevegelser.

Mer spesifikt foreslår denne oppgaven to metoder som begge benytter Gaussiske Prosesser til å lære bevegelsesmønsteret til skip i ulike scenarier basert på historiske data. Motivasjonen bak bruken av Gaussiske Prosesser er basert på dens intuitive tolkning som en statistisk fordeling over funksjoner. En slik representasjon kan dermed naturlig innlemme usikkerhet knyttet til prediksjonene som en sentral del av modellen. Et Bayesiansk statistisk rammeverk brukes i tråd med Gaussiske Prosesser for å eksplisitt vurdere den underliggende usikkerheten.

Den første foreslåtte metoden bruker et rammeverk basert på Gaussiske Prosesser direkte for å modellere posisjon i banen som en funksjon av tid. Denne tilnærmingen fungerer rimelig bra, bortsett fra i nærvær av forgrenede trafikkfelt. Formuleringen av metoden gjør strenge antakelser om unimodalitet og er ikke i stand til å representere noen form for multimodal usikkerhet.

Som en mer indirekte tilnærming forsøker den andre metoden å bruke en Gaussisk Prosess til å beskrive en latent bevegelsesmodell og bruke den til å simulere baner numerisk. Denne formuleringen er langt mer fleksibel og er i teorien i stand til å uttrykke multimodale fordelinger for de predikerte banene. Å kombinere denne tilnærmingen med et prediksjonssystem basert på et Utvidet Kalman Filter (EKF) for å simulere baner fungerer bra så lenge banene er tilstrekkelig glatte, slik at en Taylor-approksimasjon av bevegelsesmodellen fungerer som en rimelig tilnærming. Disse antagelsene gjør imidlertid denne metoden mer skjør enn den første metoden.

Begge metodene testes grundig på et reelt AIS-datasett samlet fra Trondheimsfjorden i løpet av ett år, og den statistiske ytelsen til begge metodene sammenlignes. Konsistensen av usikkerhetsestimatene blir også testet for å undersøke om metodene er i stand til å nøyaktig kunne representere den underliggende usikkerheten.





