\chapter{Conclusion}\label{chap:conclusion}
The \acrshort{gp} framework utilized in this thesis yields a powerful way of expressing beliefs about likely future trajectories. The \acrshort{gp}'s interpretation as a statistical distribution over functions allows the combination of prior knowledge and data to be used for learning complex relationships while also expressing uncertainty. Furthermore, the Bayesian framework enables these methods to incorporate expert knowledge as prior beliefs and can be considered a mix of model-based and data-driven methods. 


Two different formulations using \acrshort{gp}s are proposed. The first method directly applies the \acrshort{gp} framework to model the trajectory as a function of time. It works well and yields good results during statistical testing. However, this formulation is limited to unimodal trajectory distributions and cannot express multimodal beliefs about branching traffic lanes. 

The second method takes an indirect approach and instead uses a \acrshort{gp} to learn an unknown motion model expressed as a vector-field. This motion model is flexible enough to express multimodal distributions, though at the cost of higher complexity. This method is then paired with an EKF prediction to simulate trajectories by assuming that the distribution is Gaussian and linearizing the motion model. This GP-EKF approach works reasonably well, but it is sensitive to the choice of parameters and numerical instabilities.

Finally, an alternative solution using Sequential Monte-Carlo is discussed briefly, as the assumptions made by the GP-EKF limit the motion model's ability to express multimodal trajectory distributions.  

As the GP-EKF is a purely open-loop prediction based on gradients only, two update procedures are proposed to serve as weak feedback in an attempt to improve the robustness of the GP-EKF. However, both the \acrshort{sl} and the \acrshort{pdaf} update steps introduced in this thesis cause a significant increase in overconfident predictions and are not considered satisfactory solutions. 

The statistical testing showcases the benefit of using \acrshort{gp}s over the simpler \acrshort{cvm}. For example, on curved trajectories, the \acrshort{gp}-based methods yield far better estimates with both lower error and less variability when compared to the \acrshort{cvm}. However, for straight-line trajectories, there is still a way to go as the \acrshort{gp}-based methods perform slightly worse than the \acrshort{cvm}. The results are also affected by choice of parameters, and they are primarily intended to showcase that using \acrshort{gp}s for long-term trajectory prediction is indeed a viable solution.

The implementation used in this thesis still suffers from numerical instabilities and bad local optima during hyperparameter optimization. There is still more research required to understand how model selection can be performed more robustly. 

While there still are some quirks, this thesis has hopefully showcased the power of using \acrshort{gp}s for long-term trajectory prediction. The \acrshort{gp} framework is a compelling method, and the formulations proposed in this thesis are merely examples of how \acrshort{gp}s can be utilized. Especially the motion model used by GP-EKF is highly flexible, and the simple EKF prediction scheme used in this thesis artificially limits the true power of this method. Relaxing these assumptions, such as by using Sequential Monte-Carlo, is a promising extension to this work. 


\section{Future Work}
There are several possible extensions to the GP-EKF:
\begin{enumerate}
    \item Allow independent kernels for each output dimension.
    \item Combine \acrshort{gp} approximations with the GP-EKF to use as much of the available data as possible.
    \item Reduce the computational complexity of the GP-EKF by speeding up the hyperparameter optimization.
    \item Use a more informative prior for the motion model. As an example, the \acrshort{cvm} could be used as the mean function and yield a constant velocity estimate in areas with low data.
    \item Relax the GP-EKF assumptions in order to improve the uncertainty estimates. Using the motion model in a Sequential Monte-Carlo context may yield significantly better estimates, as it avoids linearization as well as the assumption of a Gaussian posterior distribution.
    \item Develop a more realistic likelihood for the GP-EKF update procedure. Neither the \acrshort{sl} nor the \acrshort{pdaf} introduced in this thesis accurately represent the possibility that the vessel is following the wrong trajectory and that the measurements might be all wrong. For the \acrshort{pdaf} it might be good to look into a more elaborate clutter model and not assume a fixed clutter rate.  
\end{enumerate}
